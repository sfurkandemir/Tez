%%%%%%%%%%%%%%%%%%%%%%%%%%%%%%%%%%%%%%%%%%%%%%%%%%%%%%%%%%%%%%%%%
\chapter{ALERJİK HASTALIKLAR}
%%%%%%%%%%%%%%%%%%%%%%%%%%%%%%%%%%%%%%%%%%%%%%%%%%%%%%%%%%%%%%%%%
Aşırı duyarlılık reaksiyonları Coombs ve Gell’e göre , 4 grupta toplanmıştır.
  TipI, anafilaksi, dakikalar içinde ortaya çıkan abartılmış immun reaksiyondur. En hızlı gelişen tiptir. 5-10 dakikada başlar. IgE tipi spesifik antikorları oluşur,  kana geçer,  mast hücreleri ve bazofil üzerindeki yüksek afiniteli reseptör FceRI’ya bağlanır, histamin salınımı olur. 
\begin{itemize}
   \item IgE aracılı aşırı duyarlılık reaksiyonu, bronkokonstrüksiyon, solunumda derinliğin azalmasına, burun akıntısına ve göz kaşıntısına sebep olur. Kontakt ürtiker ya da generalize ürtiker de gelişebilir. Bağırsak kasları kasılır, karın ağrısı, bulantı kusma olur. Hipotansiyon ve dolaşım yetersizliği görülür. [1]
  TipII,  sitotoksik tip aşırı duyarlılık reaksiyonu, akuttur ve antijene karşı verilen sistemik antikor saldırısıdır.  IgG , IgM  oluşumuyla antikor hücre yüzeyi reseptörüne bağlanır, ya hücre hasarı oluşur yada sito-stimülan etki oluşur.
   \item Kan transfüzyonu ile oluşan uyumsuzluk, yenidoğan hemolitik hastalığı, otoimmün hemolitik anemi, Goodpasture Sendromu örnektir.
örnektir.
  TipIII, immun kompleks tipi, soluble(çözünür) antijenle ona karşı oluşan antikor birleşip kompleman sistemini aktifleştirir. Böylece reaksiyon zinciri başlar. Şiddeti reaksiyonun dağılım ve büyüklüğüne bağlıdır.Kompleman yıkımı sonucu  C3a,C4a,C5a dediğimiz anaflatoksinler ,bölgesel olarak mast hücrenin sitoplazmasının degranülasyonu ve buna bağlı olarak damarlarda geçirgenlik artışına sebep olur. Vasküler permeabilitenin artışıyla nötrofillerin bölgeye transferi sağlanır. Ödem ve eritem bulguları vardır. Bu zincirleme reaksiyon lokal veya sistemik olabilir.  Lokalse Arthus reaksiyonu da denir.
   \item Tip III hipersensitivite reaksiyonuna örnek hastalıklar;
sistemik lupus, poliarteritis nodosa, post-streptokokkal glomerülonefrit ve en önemlisi serum hastalığıdır.
  Tip IV, gecikmiş tip hipersensitivite reaksiyonları, selüler yani hücresel immünite reaksiyonlarıyla ilgilidir. T- lenfositler, sitokinler ve aktive makrofajlar görev alır.
  \item Tip IV hipersensitivite reaksiyonuna örnek hastalıklar.;
Th1 aracılı: tüberkülin reaksiyonu, kontakt dermatit, romatoid artrit
;Th2 aracılı: alerjik rinit ve astım, atopik dermatit; Sitotoksik T lenfositler aracılı: graft reddi, tip I diabet
\end{itemize} 

 
\section{Alerjik Reaksiyonların Endodontideki Yeri}

Antijenler, mineralize dentin dışında konak doku hücreleriyle temasa geçerse alerjik reaksiyonlar meydana gelir. Yumuşak doku kontaktı, endodontide apeks veya yan-aksesuar kanallarla, furkasyon bölgesinden olabilir. Retrograd veya taşkın dolgular, temas alanını belirler ve böylece reaksiyon şiddetini de etkiler. Endodontik  dolgularda sızıntı olursa dentin tübülleri aracılığıyla sıvı geçişi çok önemlidir. Eğer iyi mineralize aselüler bir sement tabakası varsa çok iyi bir bariyer oluşturur, tübüllerden sıvı transportu engellenir.
Bölgenin anatomisi, tübüllerin dizilişi, yan kanallar  çeşitlilik gösterdiği için potansiyel alerjenler ile oluşacak doku teması riski artmaktadır.\\

POTANSİYEL ALERJENLER;
\begin{itemize}
   \item LATEKS (ELDİVEN, RUBBER DAM LASTİĞİ)
   \item DEZENFEKTAN VE İRRİGASYON SOLÜSYONLARI (SODYUM HİPOKLORİT, OKSİJENLİ SU..)
   \item LOKAL ANESTEZİ
   \item DOLGU MATERYALLERİ (KÖK KANAL PATI,GEÇİCİ DOLGU VE İLAÇLAR, APİKAL DOLGULAR..)
   \item ANTİBİYOTİKLER
   \item METAL
\end{itemize}



\subsection{Latekse Alerji}
Kauçuk içeren lateks, günümüzde bir çok alanda karşımıza çıkmaktadır.  Diş hekimliğinde eldiven, tükürük emici, plastik şırınga, rubber dam, tedavi sırasında çocukların temas ettiği ürünlerin yanı sıra; biberon, balon gibi günlük hayattaki malzemelerde de karşımıza çıkmaktadır. Dental işlemler sırasında en çok karşılaşılan alerji tipidir. Alerji bulguları subklinikten ağır sistemik reaksiyonlara kadar değişkenlik gösterir.
Mukozal temas, deri temasından daha şiddetli sonuçlar doğurur.  Lateks proteinleri, IgE bazlı kontakt dermatit reaksiyonu ortaya çıkarır (Tip I).  İşlem yapılan bölge oral bölge olduğu için, etkilenen bölge larenks, farenks, üst solunum yolu bölgeleridir. Bölge önceden uyarılmışsa anafilaktik semptomlar görülebilir. Üretim hatalarına bağlı olarak gecikmiş dermal veya mukozal reaksiyonlar oluşabilir.
\%29un üzerinde diş hekimi kullandığı 200 çeşit lateks eldiven ile ilgili alerjik reaksiyon raporu sunmuştur. Kontakt dermatit şikayeti zamanla yerini “gerçek alerji”ye bırakmıştır.(7) Tüm bunlar tip I ve tip IV reaksiyonlardır. Tip I daha nadir görülmektedir.
Bir olgu raporuna göre.(7)  9 yaşında, bayan bir hasta, annamnez sırasında annesinin aklına lateks alerjisi olduğu gelmiştir.  Ilk temas doğumgününde balonlara üflerken olmuştur.  Sonrasında da eldiven ve elastic bantlar la teması olduğunda aynı olaylar yaşanmıştır. Annesinin dediğine göre laboratuvar testi yapılana kadar 3 büyük reaksiyon vermiştir. Bunun yanında hastanın asetaminofen ve aspirin kullanması sakıncalı ve toza alerjisi ve bronşiti vardır. Anamnez alındıktan sonra hekim metal enjektör ve alternatif rubber dam kullanmıştır. Lateks içerikli herhangi bir materyal kullanılmamıştır.
Bir diğer olgu raporuna göre.(7) 5 yaşında, bayan bir hasta, çiğneme sırasında diş ağrısı şikayetiyle başvurmuştur. Anamnezde balon temasında ciltte erüpsiyon ve ürtikerler oluştuğu saptanmıştır. Ancak laboratuvar testleri negative göstermektedir.Yine de vinil eldiven kullanılmıştır.
Bu iki vakada da antihistaminik ya da kortikosteroid reçetelenmemiştir çünkü tip I reaksiyon gözlenmemiştir. Eğer lateksle kontakt engellenemez durumdaysa bir profilaksi yapılması ciddiyeti azaltabilir.
Mehra ve Hunter’ın araştırmalarına göre.(7) spina bifida, ürolojik anomaliler,premature doğum, mental retardasyon, serebral paralazi olan hastalar çok fazla operaston geçirdiği için latexe alışmışlardır ve daha duyarlılardır. Bunun yanı sıra rinit, astım, yemek alerjisi olanlarda ellerde dermatit, belirli yiyecek alerjisi  lateksle birlikte çok daha ciddidir.
Bu hastalardan çok iyi bir anamnez alınmalı ve ebeveynlerden allerjiyle alakalı her şey sorgulanmalıdır. (ilaç, lateks, yiyecek-içecek,kivi,fıstık,domates,muz, gibi..) Gerekirse in vitro test spesifik anti doğal kauçuk latexg E testi istenmelidir.
İrritan dermatit ve aşırı duyarlılık tip IV’ün endikasyonu sonrası;
Hekimler lateks içerikli hiçbir material kullanmamalıdır.  Temas direkt kesilmelidir. Tip I hiper sensitivite reaksiyonu çok daha ciddidir. Birden alevlenir.
Non-lateks eldiven kullanımıyla çözülür. Vinil, neopen,neolon, nitril veya polimer  eldivenler piyasada mevcuttur.
Rubber dam kullanırken lastik, alternatif olarak vinil (flexi) dam olmalıdır.   Tüm bu önlemlere rağmen hastada gecikmiş tip reaksiyon görülmeyeceği garanti değildir. Bölgede eritem, şişlik, dilde ülserafif lezyonların görüldüğü araştırmalar mevcuttur.
Eğer müdahale sırasında reaksiyon oluşursa antihistamin ve kortikosteroid ile tedavi edilip hayat kurtarılabilir.
Hastada astım krizi de varsa bronkodilatör bir antiastmatik verilmelidir.





\subsection{Dezenfektan Ve İrrigasyon Solüsyonlarına Alerji}
Endodontik tedavide amacımız ortamdaki bakterileri etkisiz hale getirip minimuma indirmek olduğu için doğaldır ki dezenfektan ve antibakteriyellerden yaralanırız. Dentin kanalları, aksesuar kanallar, kanal ramifikasyonları, apikal deltalar ve transvers anastomozlar gibi kompleks bölgelere[4] yerleşen bakterileri mekanik preparasyonun yanında irrigasyon solüsyonlarıyla da elimine edilmektedir. Bu solüsyonlar seans sırası ve seanslar arası da kullanılmaktadır.

Kök kanal irrigasyon solüsyonu olarak en sık asitler(sitrik ve fosforik asid), şelasyon ajanları (EDTA),proteolitik enzimler, alkalen solüsyonları (sodyumhipoklorid, sodyum hidroksid, glioksid), lokal anestezikler ve salin kullanılmaktadır.[4] Endodontide kullanılan içeriği fenol, formokresol kresatin, formaldehitle birlikte olanlar alerjik maddelerken, hidrojen peroksit kalsiyum hidroksit alerjik maddeler değildir.[2]

\begin{itemize}
   \item SODYUM HİPOKLORİT;\\
   En çok kullanılan irrigasyon solüsyonudur. Bakterilere, bakteriyofajlara, sporlara, mayalara ve virüslere karşı etkinliği kanıtlanmış geniş spektrumlu antimikrobiyal bir ajandır. Canlı dokuya çok düşük dozda çok fazla toksik etkisi vardır. NaOCl ne kadar düşük dozdaysa o kadar az toksiktir, minimum seviyede tutulmalıdır.
Siquera ve ark. kontamine kanallı çekilmiş dişlerde \%1’lik, \%2.5’luk ve \%5.2’lik NaOCl uygulamaları arasında antibakteriyel etkinlik açısından herhangi bir farklılık bulamamışlardır.(2)
Berber ve ark. NaOCl’in çeşitli konsantrasyonlarının ve preparasyon tekniklerinin kök kanallarında ve dentin tübüllerinde E.faecalis miktarını azaltmadaki etkinliğini araştırmışlar, kök kanallarının dezenfeksiyonunda konsantrasyonlar arasında bir farklılık bulamamışlardır.(4)

NaOCl, proteolitik etkili olduğu için inorganik kalıntılar serbest klor miktarına bağlı olduğundan yüksek konsantrasyonda ve az miktarda kullanılması düşük konsantrasyonda ve fazla uygulanmasıyla denktir.

Çalışkan ve ark., endodontik tedavi esnasında \%1 NaOCl irrigasyonu yapıldığı anda şiddetli ağrı ve yanma, yanak ve dudakta şişlik meydana gelen bir vaka bildirmişlerdir. Birkaç dakika sonra hasta nefes alma zorluğu yaşamıştır. Deksamethason ve antihistaminik enjeksiyonu yapılmıştır. Bu reaksiyonlar, sodyum hipokloritin yumuşak dokulara enjeksiyonu sonucu yada alerji sonucu gelişmiş olabileceği de düşünülmüştür. Hasta anamnezinde evde çamaşır suyu kullanımı esnasında da ellerinde kızarıklık ve nefes alma güçlüğü çektiğini bildirmiştir..(2)
Serper ve ark.,endodontik tedavi esnasında rubber-dam’dan sodyum hipoklorit sızması sonucu ciltte yaralanma meydana gelen bir vaka bildirmişlerdir. Endodontik irrigasyon esnasında yanma şikayeti bildiren hastanın çenesinde döküntüler meydana geldiği rapor edilmiştir.(2)
Keçeci ve ark. sodyum hipoklorit irrigasyonu esnasında apikalden taşma sonucu şiddetli ağrı ve şişlik oluşan iki vaka bildirmişlerdir. (2)
Bu hastalardan iyi bir anamnez alınmalı, allerji durumu ve çamaşır suyu kullanımı sorgulanmalıdır.
Iyi bir izolasyonla çalışılmalı, rubber dam’dan irrigasyon solüsyonu taşmamalıdır.
Kök kanalında irrigasyon iğnesiyle apexten 1.5-2 mm geride çalışılmalıdır.
Önlem alınmazsa, kimyasal yanıklar ve doku nekrozları, nörolojik komplikasyonlar, üst solunum yolu obstrüksiyonu oluşabilir. (5)

   
   
   \item KLORHEKSİDİN GLUKONAT; \\
   Plak önleyici olduğu için çürük oluşumu önlenir. Periodontolojide temel amaç plak önlemek olduğundan tedavide sık kullanılır.
Hücre duvarına absorbe olan mikroorganizmaların membran bütünlüğünü bozar ve intrasellüler komponentler sızar. İçeri sızan pozitif yüklü iyonlar dentine tutunur ve dentin yüzeyinde mikroorganizmaların koloni oluşturması engellenir.
\%0.2’lik düşük konsantrasyonlarda bakteriyostatik, \%2’lik yüksek konsantrasyonlarda bakteriyosidal etki göstermektedir.(4) Klorheksidinin jel ve likid formunun tüm konsantrasyonlarının (\%0.2’lik, \%1’lik, \%2’lik) test edildiği bir çalışmada E.faecalis’in eliminasyonunda \%5.2’lik NaOCl kadar etkili olduğu bildirilmiştir. Ancak Estrela ve ark. her iki irrigasyon solüsyonunun E.faecalis’in eliminasyonunda düşük etki gösterdiğini bildirmişlerdir. (4)
Klorheksidin düşük toksisitede etkili bir antimikrobiyal ajandır ancak dezavantajı doku artıklarını çözme yeteneğinin olmamasıdır. Klorheksidin ve NaOCl’in kombinasyonuyla bu dezavantajı gidermeye çalışılmıştır. Ancak bu kombinasyonun iyi bir antimikrobiyal ve doku çözücü özelliğe sahip olmakla birlikte, smear tabakasını uzaklaştırmada yetersiz kaldığı bildirilmiştir. Klorheksidin ve NaOCl’in birlikte kullanımının dişlerde renk değişikliğine yol açtığı ve çökelti oluşturduğu saptanmıştır. Yapılan bir in vitro çalışmada bu iki irrigasyon solüsyonunun kombine kullanımının E.faecalis’e karşı klorheksidinin tek başına kullanımından daha etkili olmadığı bildirilmiştir . (4)

Klorheksidinin prospektüsüne bakıldığında, baş dönmesi, bulantı gibi yan etkisi bulunmasına rağmen herhangi bir alerjik reaksiyon normal kullanım koşullarında saptanmamıştır. Ancak, Hatipoğlu H. ve ark. (10) çalışmalarında hatalı kullanım sonucu ağız mukozasında deskuamatif lezyonlar görüldüğünü saptamışlardır. 41 yaşındaki bayan hasta, günde 15 dakika klorheksidinli gargara ile ağzını çalkalamıştır. Sonrasında ağız içinde ki lezyonlarla periodontoloji kliniğine başvurmuştur.Gargara kesildikten 1 hafta sonra lezyonlar da kaybolmuştur.

       

Piyasada Klorhex®, Kloroben®, Andorex® ticari isimleriyle klorheksidin glukonat etken maddeli gargaralar mevcuttur.  Özellikle periodontolojide çok kullanılan Kloroben gargaradır. Cerrahide profilaksi amaçlı da kullanılmaktadır.  Endodontide ise kök kanal irriganı olarak kullanılmaktadır.
Hasta anamnezi ile alerji sorgulanmalıdır. Aspirasyon iyi yapıldığı sürece alerji riski minimumdur.
Rubber dam takmak da irrigasyon sırasında oluşabilecek kazaları önler.

   
   \item ETİLEN DİAMİN TETRA ASETİK ASİT;\\
   EDTA, ilk kez 1957 yılında Nygaard-Østby tarafından dar ve kalsifik kanalların preparasyonuna yardımcı olması amacıyla endodontide kullanılmaya başlanmıştır.(4)Kök kanalında dentini kimyasal olarak yumuşatır, dentin artığı olan smear tabakasını uzaklaştırır ve dentinin geçirgenliğini böylece artırır. EDTA’nın \%15-17 arası konsantrasyonlarda kullanılması önerilmektedir.(4) Smearda EDTA inorganik, NaOCl organik yapıyı uzaklaştırır, bu yüzden ikisi birlikte kullanılırsa etkili bir çözüm olur. E.faecalis’in eliminasyonunda NaOCl ve EDTA’nın kombine kullanımının NaOCl’in ardından EDTA kullanımından daha etkili olduğu bildirilmiştir. (2) Mineral içeriğinde değişiklik yaparak hem tübüllerde hem de apikal bölgede etkili temizlik sağlanır. Tarama elektron mikroskobunda (SEM) yapılan bir çalışmada \%2.5’luk NaOCl ile \%2’lik klorheksidin jel veya likid formunun ardından EDTA ve salin uygulaması ile kök kanal duvarlarında daha temiz bir yüzey elde edilmiştir . (4)
EDTA  endodontik tedavide kullanıldığı kadarıyla alerjik reaksiyona sebep olmamıştır. Lokal olarak sert dokuya uygulanan bir şelasyon ajanıdır. Kök kanal boyunda çalışıldığı sürece alerjik reaksiyon vermez.
 \item OKSİJENLİ SU;\\
 Oksijenli su çok iyi bir antibakteriyeldir.
Kök kanal tedavisinin başarısı için amaç, kök kanalından mikroorganizmaların yok edilmesidir. Şekillendirmede kullanılan materyaller, irrigasyon malzemesi, seans arası kanal içine konulan materyalin miktarı da önemlidir.(4)
Oksijenli su, aslında çok toksik olan NaOCl irriganı yerine kullanılan önlem niteliğinde bir materyaldir. Kök kanal irrigasyonu işleminde her zaman iyi aspirasyon yapılmalıdır. Rubber dam takmak da irrigasyon sırasında oluşabilecek kazaları önler.

   
\end{itemize}



\subsection{Lokal Anestezik Maddelere Alerji}
Diş hekimliğinde kullanılan mevcut lokal anestezikler genellikle iyi tolere edilirler. Nadir de olsa alerjik reaksiyon verilebilir. 
Genellikle santral sinir sistemi ve kardiyovasküler sistemle kombine görülen reaksiyonlardır. Bu bölgelerde toksik etki ile delirium, konvülsiyon, hiperventilasyon, vasovagal senkop, endojen sempatik stimülasyon oluşuyorken, nabız artışı, hipotansiyon ve senkop ile kardiyovasküler etkiler görülmektedir. Astım da alerjik reaksiyon verilmesine sebep olur. Bunlar böylece toksik, kardiyovasküler ve astıma bağlı olmak üzere Gilman tarafından 3 gruba ayrılmıştır. (1)(11) Bunlar psikomotor ve operatif travma gibi lokal anestezikten bağımsız yanıtlardır.(1)  Yine de en çok korkulan ve tehlikeli olan anafilaksi, %0.5 (1) görülür. 
Diş hekiminin  “Daha önce dişinize iğne yapıldı mı?” “Diş çekimi yapıldı mı?” “Herhangi bir şeye alerjiniz var mı?” gibi sorularının anamnezde sorulması hastanın daha önce lokal anesteziye reaksiyon verip vermediğini öğrenmemize olanak sağlar. Hasta daha önce lokal anestezi enjeksiyonuyla reaksiyon vermişse ve bir daha o ilaç türünü ömür boyu kullanmaması söylenmişse bile; bu tip alerji çok nadir görülmektedir. Diş hekimi tehlikenin bilincinde olmalı şüpheli durumda epikütan testi yapılabilir. 


\subsection{Dolgu Materyallerine Alerji}
\subsubsection{Geçici dolgu materyallerine alerji}
Tek seansta tamamlanamayan kanal tedavisi seans aralarında pulpa boşluğunu kapatmak için kullandığımız geçici dolgu materyallerinin oral mukoza ile muhtemel teması sonucu alerjik reaksiyonlar verebilir. 
Özellikle mukozayla en çok temas edebilecek olan  öjenol, fenol bileşiği karanfil yağıdır. Diş hekimliğinde tedavide amaç sızdırmazlık özelliğini kullanarak derin çürük kavitelerde iyileştirici gücünü göstermesidir.  Örtücü özelliği ile çiğneme kuvvetlerine dayanıklıdır. Kimyasal ve fiziksel etkilere dayanıklı ancak diş hekimi müdahalesi sonucu kolay sökülebilir materyaldir. Dezavantajı ise dokuyla temas anında çözünmesi ve absorbsiyonu çok çabuk olur. Alerjik ve toksik etkisi dolayısıyla ürtikerlere sebep olabilir. En yaygın kullanılan IRM, Cavit ve Coltosol’dür.




\subsubsection{Kök kanal patları ve dolgularına alerji}
Kanal tedavisinin tamamlayabilmek için bakteriden elimine edilmiş kök kanalının doldurulması gerekir.  Bunun için kullanılan materyaller; katı ve siman/pat olarak 2 grupta incelenir. Katı olanlar; güta perka, gümüş kon ve resilon’dur. Siman ve patlar grubunda; ZnOE, CaOH, kloroperka,formaldehit bulunmaktadır. Rezin içeren patlar ise cam iyonomer, silikon esaslı pat ve bioseramiktir. 
Retrograd dolgu olarak ise amalgam, güta perka, cam iyonomer siman, rezin modifiye cam iyonomer siman, ZnOE, kalsiyum sülfat, Mineral Trioksit Agregat kullanılabilir. Bu materyallerden bilinen alerjenler amalgam, çinkooksit öjenol simanı ve rezin bazlı materyallerdir.(1)\\

KOR MADDELERE ALERJİ

\begin{itemize}
   \item Güta Perka'ya alerji\\
   Güta perka, kök kanal dolgusu olarak en yaygın kullanılan materyaldir. Hammaddesi kauçuktur. Fiziksel özellikleri (sert, kırılgan ve esnek) ve kimyasal özellikleri iyidir. Güta perkanın kimyasal yapısı rubber dam ile benzemektedir. Bu yüzden lateks alerjisi olan hastalarda güta perka kaynaklı kontakt dermatit oluşması muhtemeldir.(12)  Ancak güta perka tek başına alerjik bir madde olarak kabul edilmemektedir.(1)  Alerjik reaksiyon oluşması için gütanın apexten taşmış olması gerekmektedir.
   
   \item Gümüş Kon'a Alerji\\
   1960’lardan beri kullanımını yitirmiş olan bu yöntem, özellikle dar çaplı kanallarda gümüşten bir malzemenin dolgu maddesi olarak kullanılmasıdır. Metale alerjisi olan hastalarda kanal boyu aşılmış bir dolum yapılırsa alerjik reaksiyon muhtemeldir.
   
   \item Resilon'a Alerji\\
   Bilindiği üzere rezin maddesi diş hekimliğinde adeziv bağlantı oluşumuna olanak tanımaktadır. Resilon ise pat ve kor arasında bağlantıyı sağlamaktadır. Epiphany materyali ise resilon ile kullanılınca kökteki dentin dokusuna bağlanmaktadır.  Yani kompozit dolgularda yaptığımız bonding işlemindeki kullanılan material kök kanal dolgusunda epiphany adını almaktadır. Güta perka ile resilon kombine kullanılmaktadır. Kondensasyon teknikleri veya kanal içi enjeksiyon ile ulaşım sağlanabilir. 
 
\end{itemize}

SİMAN VE PATLARA ALERJİ
\\
Kök kanal tedavisinin son aşaması olan kanal dolumunda katı materyal, kanala fiziki destek sağlasa da tam dolumu sağlamaz. Akışkan bir materyale daha ihtiyaç duyarız. Kanal şekillendirme sırasında kök boşluğunda düzensizliklerde, aksesuar kanal varsa, kor materyali olan güta perkalar konik şekillidir ve aralarında boşluk vardır. Tüm bu buşluklar akışkan bir madde ile kolayca kapatılabilir.  Bunun için ise radyoopak, boyutsal stabilitesi olan, bakteostatik, doku sıvılarında çözünmeyen ancak retreatment’e olanak sağlayacak olan patlar kullanılmaktadır.  Günümüzde ideal pata ulaşamamış olsak da avantajları olanlar vardır. Akışkan olduğu için özellikle mikromotor ile uygulanan patların apexten taşması kuvvetli ihtimaldir. Bu da alerjik reaksiyona sebep olabilir.

\begin{itemize}
   \item ZnOE içerikli Patlara Alerji\\
   “Geçici Dolgu Maddelerine Alerji” bölümünde anlatılanlara ek olarak ;
Endodontide ise antimikrobial özelliğinden yararlanılarak, kanal patı ve geçici dolgu materyali olarak kullanılmaktadır.  Başarılı tedaviler sağlamaktadır. Minimum düzeyde polimerizasyon büzülmesine uğrar. Kanal patında kullanımında likid kısım öjenoldür ve tozun akıcılığını artırır. Lokal irritasyon, sitotoksik reaksiyon ve hipersensitivite reaksiyonlarına sebep olabilir.(16)

   \item Kalsiyum Hidroksitli Patlara Alerji\\
   Bazik özelliği ile antimikrobialdir. Kuafaj olarak kullanılırsa  apeksogenezis sağlanır. Endodontide genellikle seans aralarında kanalı geçici olarak doldurmada kullanılır. Olgu raporuna göre 21 yaşında kadın hastaya 25 nolu dişin kanalına uygulanmasının ardından, ilgili dişin apeksinde düzensiz nekroze bölge gözlenmiştir. Bu da yine apeksten taşkın dolgu sebebiyle olmuştur. (17)
   \item Formaldehit İçerikli Patlara Alerji\\
   Formaldehitin  antimikrobial bir maddedir. Patlara paraformaldehit eklenerek bu özellik kazandırılır. ZnOE ile kombine kullanılır.  Rezin ile kombine kullanılanlar, öjenollülere göre daha az formaldehit salarlar. Yapılan araştırmada(19) 67 yaşında bir hastanın kök kanal dolumu formaldehitli pat ile yapılmıştır. RAST testi sonucu formaldehit pozitifken deri pikür testi negatif çıkmıştır. Duyarlılık 1 yıl önceden ortaya çıkmış gibi görünmektedir. Diş tedavisi sırasında komplikasyon verebilir ve anafilaksiye kadar gidebilir. Riskli hastalarda spesifik IgE değerlendirilmesi düşünülmelidir. (18) Çünkü formaldehite karşı gelişen reaksiyon tip 1 alerji reaksiyonudur. Anti-formaldehit IgE artışı vardır. 4 vakadan 2’sinde anafilaktik şok 2’sinde  ürtiker oluşmuştur.(1)
   \item Rezin içerikli Patlara Alerji\\
   “Resilon’a Alerji”e ek olarak;
Dentin adeziv materyaldir ayrıca öjenollü değildir.  Alerjen madde bisfenol-A-diglisidil eterdir, ayrıca gümüş, bizmut da bulunur. Poliketon (polivinil) içeriyorsa alerjen madde propioilasetofenon monomeri kaynaklı diklorofendir. (1) Metakrilat, kontakt dermatit ve astıma sebep olabilir.(18) Çalışmada likenoid reaksiyon veren ve pikür testi kompozite pozitif olan hastalarla çalışılmıştır. Antifungal tedavisiyle gelişme görülmüştür. Başka bir çalışmada(18) ise rezin bazlı kanal dolgusu yapılan hastalardan alerjik kontakt stomatit, intraoral bölgede eritema görülen hastalar araştırılmış ve pikür testi Bis-GMA’ya pozitif çıkan yalnızca 2 kişi olmuştur.(18)  Bunun sebebi ise yine Formaldehit İçerikli Patlara Alerji  başlığınki gibi alerjen materyalle karşılaşımamasıdır. IgE reaksiyonlarından en az 1 yıl önce maddeyle karşılaşılması gerekliliğidir. 

\item Cam İyonomer içerikli Patlara Alerji\\
Rezinler gibi dentine bağlanma özelliği vardır ve sadece kanal dolgusu için değil kanal dolumu tamamlandıktan sonra geçici dolgu olarak kullanılabilirler. Ketac-Endo, kanal patı olarak kullanılabilmektedir. Toksisitesi düşük olduğu için minimal reaksiyonlar vermektedir. Avantajı adezyon sağlaması ve allerjen özelliğinin az olmasıyken; dezavantajı antimikrobial özelliğinin olmaması ve retreatmenta olanak vermemesidir. Ancak yine de apeksten taşmamaya dikkat edilmelidir. Direkt pulpa kuafajı sonucu mental sinir parestezisi gelişen vaka sunulmuştur. (1)

\item Slikon İçerikli Patlara Alerji\\
Bu preparatlara örnek olarak RockoSeal vardır. PDMS (polidimetilsiloksan) esaslıdır. Avantajı, sızdırmazlık ve sitotoksisitesinin az olmasıyken dezavantajı çözünürlüğünün iyi olmasıdır. Sertleşirken genleştiği için apeksten taşma ihtimali vardır. 

\item Kalsiyum Silikat Patlara Alerji\\
MTA bu preparatlara örnektir. Genelde retrograde dolguda kullanılır. Hidrofilik bir toz olan trikalsiyum sülfat kaynaklıdır. Radyoopaklığı sağlayan bizmut oksit ve  dişi renkleştirmemesi için zirkonyum dioksit bulunur. Alerjik reaksiyon vermemiştir. Dikkat edilmesi gereken, özellikle retrograd dolgu olarak kullanıldıktan sonra o bölgenin resumla yıkanmamasıdır yoksa mikrosızıntı oluşur. Tedavi başarısız olur. 
   
\end{itemize}

RETROGRAD DOLGU MATERYALLERİNE ALERJİ \\
Kanal dolgu maddelerinin özelliklerine ek olarak ,sementogenezisi uyarmalı, elektrokimyasal olarak İnaktif olmalı ve dolayısıyla korozyona uğramamalıdır. ZnOE’nin IRM ve SuperEBA preparatları, cam iyonomer siman, resin kompozit’in Retroplast preparatı, resin modifiye cam iyonomerin Geristore preparatı ve MTA’nın yanında amalgam da retrograde kullanılabilmektedir.

\begin{itemize}
   \item ZnOE Retrograd Dolgusuna Alerji \\
   “ Geçici Dolgu Materyallerine Alerji ve ZnOE İçerikli Patlara Alerji konularında anlatılmıştır. Ancak retrograd dolgunun alerji yaratması tedavide tamamen  başarısızlıktır. Immün sistem devreye girdiği için direkt kemik nekrozuna gider.  
   
   \item Amalgamın Retrograd Dolgusunda Alerji\\
   Amalgam alerjik reaksiyon verirse eritematöz ve likenoid lezyonlar  deri, oral mukoza,yüz ve boyunda görülebilir.(18) Genellikle civa alerjisi olanlarda görülür. Dental operasyon ortamında civa içerikli her şey elimine edilmelidir. Ultrasonik aletle adaptasyon yerine konvansiyonel yöntem tercih edilmelidir. Operasyon sahasına yanlışlıkla partikül sıçramamalı, amalgam düşürülmemelidir. Amalgam restorasyon oral mukozada ilgili bölgede renkleşme yapabilir.
 
\end{itemize}



\subsection{Metallere Alerji}
Civa (Hg), gümüş (Ag), altın (Au), kalay (Sn), bakır (Cu), çinko (Zn), demir (Fe), nikel (Ni), krom (Cr), kadmiyum (Cd), platin (Pt), kobalt (Co), molibden (Mo), tantal, titanyum (Ti) gibi metaller ve alaşımları günlük hayatta da dental tedavide de karşımıza çıkabilecek alaşımlardır. Korozyon arttıkça metal iyonu açığa çıkması artar, dolayısıyla doku reaksiyonu riski de artar. Bakır, korozyonu artıran metaldir. Titanium en güçlüsüdür.
Endodontide kullanılan metaller; Ni-Ti eğeler, rubber dam çerçevesi, klempler ve forsepstir. 

\begin{itemize}
   \item Ni-Ti Eğelere Alerji \\
   Diş hekimliği malzemeleri arasında en çok duyarlılık gösterilen materyallerden biri de nikeldir. Diş hekimliğinde kullanımı dışında kanserojendir.(1) Özellikle kontakt dermatite bağlı alerji göstermektedir. Klinik seyri ise yanan ağız, gingivada hiperplazik alanlar ve dilde anestezi, şelit, stomatit olarak karşımıza çıkmaktadır, pikür testi ile kesin tanı konur.(18)
Titanyum, dental implantların temel malzemesidir, eğelerde de kullanılır. Biyouyumlu bir materyaldir. Alerjik reaksiyonlar nadiren görülür. Eğer alerji varsa ürtiker, egzama, mukozada kızartı, ödem, kardiak aritmi gibi sistemik bulgu da görülür.  Ancak bu kadar büyük reaksiyonlar genelde implantta görülmektedir. Kardiak pil kullanan hastalarda granulomatöz reaksiyonlar görülmüştür.(18)

   \item Civa'ya Alerji \\
   Civa intoksikasyonu; dişeti kanaması, alveolar kemik rezorbsiyonu, tükürük sekresyonu, halitozis, dişetlerinde anormal pigmentasyon, lökoplaki, stomatit, dişeti, damak ve dilde ülserasyonlar, yüzde ve dudaklarda karıncalanma ve yanma hissi, ağızda metalik tat gibi bulgularla ayırt edilir.(18) Diş hekimliğinde amalgam ile alınır ve amalgam da bir alaşım olduğu için bu konuya dahildir. “Amalgamın Retrograd Dolgusuhna Alerji.”  konusunda belirtilenler geçerlidir.
Cıva teması kesilmeli, etken maddeler elimine edilmeli, antioksidan tedaviyle kombine şelasyon tedavisi uygulanmalıdır. DMPS (dimerkap- tanpro- pan sülfat), DMSA(dimerkaptosüksinik asit), lipoik asit sülfidril grupları şelasyonda kullanılır.(1)

   \item Diğer Metallere Alerji\\
   Krom alerjisi nadirdir. Ancak metallere alerji kontakt dermatit sonucu oluşmaktadır. Rubber dam çerçevesi ve forseps cilde, klempler oral mukozaya temas eden çelik malzemelerdir. İyi bir anamnez alınırsa önlemler alınabilir ve plastik malzeme kullanılabilir. 
\end{itemize}



\section{Alerjik Reaksiyon Durumunda Yapılacaklar}
\begin{itemize}
   \item Acil yardım çağırılmalıdır.
   \item Hasta supin pozisyonunda yatırılmalı, ayakları kalp seviyesinin üstünde olmalıdır.
   \item Kan basıncı, kalp hızı ve ventilasyon hızı değerlendirilmelidir.
   \item Erişkinde 0.3 mg 1:100’lük adrenalin intradermal veya intramukozal verilmelidir. Hastanın bronkospazm, hipotansiyon gibi semptomları hafifleyene kadar 5 dakikada 1 uygulanabilir. Gerekirse oksijen verilir.
   \item Difenhidramin 50 mg veya klorfeniramin 10-20 mg yavaşça intravenöz enjekte edilir.
   \item Hidrokortizon 200 mg intravenöz olarak enjekte edilir.
  
\end{itemize}
